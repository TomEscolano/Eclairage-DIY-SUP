Projet de 2e année de B\+TS S\+N\+IR.



\section*{Présentation générale du système supportant le projet}

L’éclairage extérieur ouvre une vision sur le jardin en chassant les zones d’ombre, permet de mener des activités le soir sur la terrasse ou la pelouse \+: lire, faire des grillades au barbecue, prendre un repas, jouer aux cartes ou au ping-\/pong. Il est également utile pour ranger ce qui traîne dans le jardin à la nuit tombée, chercher un objet égaré, rentrer la voiture dans le garage ou mettre la clé dans la serrure. Judicieusement réparti dans le jardin, l’éclairage (blanc, unicolore ou multicolore) s’intègre dans le décor et crée une belle atmosphère nocturne en sculptant les volumes comme le démontre la figure 1. Implanté ponctuellement, il met en scène un arbre, un massif, un plan d’eau, une pergola, une particularité architecturale de la maison, une statue ou une jolie allée pavée.~\newline
 Un luminaire extérieur facilite les déplacements et limite les risques de chutes en signalant un escalier, une dénivellation ou le franchissement de sols différents, du gazon à la terrasse. Il balise une allée sinueuse, facilite la sortie des poubelles ou le franchissement du portail. Il possède également un effet dissuasif vis-\/à-\/vis des visiteurs indélicats en simulant la présence des propriétaires.

\section*{Objectif du candidat}

Vous développez le module \textquotesingle{}\hyperlink{classSUP}{S\+UP}\textquotesingle{} (Supervision des éclairages).~\newline
 Vous devez \+:~\newline
 – Développer une application W\+EB pour gérer (ajout, modification, suppression, configuration, activation, cycle) les éclairages~\newline
 – Mettre en place l’authentification d’accès à l’application W\+EB~\newline
 – Configurer via T\+C\+P/\+IP les éclairages multicolores~\newline
 – Mettre à jour les configurations des éclairages multicolores à partir des données reçue~\newline
 des contrôleurs~\newline
 – Enregistrer dans une base de données les configurations des éclairages multicolores~\newline
 – Mettre à jour l’application des éclairages multicolores~\newline
 – Commander, à distance, les éclairages~\newline
 – Exporter dans un fichier au format csv des configurations des éclairages~\newline
 – Importer à partir d’un fichier au format csv des configurations des éclairages~\newline
 